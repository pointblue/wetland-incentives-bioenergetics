\documentclass[]{article}
\usepackage{lmodern}
\usepackage{amssymb,amsmath}
\usepackage{ifxetex,ifluatex}
\usepackage{fixltx2e} % provides \textsubscript
\ifnum 0\ifxetex 1\fi\ifluatex 1\fi=0 % if pdftex
  \usepackage[T1]{fontenc}
  \usepackage[utf8]{inputenc}
\else % if luatex or xelatex
  \ifxetex
    \usepackage{mathspec}
  \else
    \usepackage{fontspec}
  \fi
  \defaultfontfeatures{Ligatures=TeX,Scale=MatchLowercase}
\fi
% use upquote if available, for straight quotes in verbatim environments
\IfFileExists{upquote.sty}{\usepackage{upquote}}{}
% use microtype if available
\IfFileExists{microtype.sty}{%
\usepackage{microtype}
\UseMicrotypeSet[protrusion]{basicmath} % disable protrusion for tt fonts
}{}
\usepackage[margin=1in]{geometry}
\usepackage{hyperref}
\hypersetup{unicode=true,
            pdftitle={OUTPUTS READ ME},
            pdfauthor={Kristen E. Dybala},
            pdfborder={0 0 0},
            breaklinks=true}
\urlstyle{same}  % don't use monospace font for urls
\usepackage{graphicx,grffile}
\makeatletter
\def\maxwidth{\ifdim\Gin@nat@width>\linewidth\linewidth\else\Gin@nat@width\fi}
\def\maxheight{\ifdim\Gin@nat@height>\textheight\textheight\else\Gin@nat@height\fi}
\makeatother
% Scale images if necessary, so that they will not overflow the page
% margins by default, and it is still possible to overwrite the defaults
% using explicit options in \includegraphics[width, height, ...]{}
\setkeys{Gin}{width=\maxwidth,height=\maxheight,keepaspectratio}
\IfFileExists{parskip.sty}{%
\usepackage{parskip}
}{% else
\setlength{\parindent}{0pt}
\setlength{\parskip}{6pt plus 2pt minus 1pt}
}
\setlength{\emergencystretch}{3em}  % prevent overfull lines
\providecommand{\tightlist}{%
  \setlength{\itemsep}{0pt}\setlength{\parskip}{0pt}}
\setcounter{secnumdepth}{0}
% Redefines (sub)paragraphs to behave more like sections
\ifx\paragraph\undefined\else
\let\oldparagraph\paragraph
\renewcommand{\paragraph}[1]{\oldparagraph{#1}\mbox{}}
\fi
\ifx\subparagraph\undefined\else
\let\oldsubparagraph\subparagraph
\renewcommand{\subparagraph}[1]{\oldsubparagraph{#1}\mbox{}}
\fi

%%% Use protect on footnotes to avoid problems with footnotes in titles
\let\rmarkdownfootnote\footnote%
\def\footnote{\protect\rmarkdownfootnote}

%%% Change title format to be more compact
\usepackage{titling}

% Create subtitle command for use in maketitle
\providecommand{\subtitle}[1]{
  \posttitle{
    \begin{center}\large#1\end{center}
    }
}

\setlength{\droptitle}{-2em}

  \title{OUTPUTS READ ME}
    \pretitle{\vspace{\droptitle}\centering\huge}
  \posttitle{\par}
    \author{Kristen E. Dybala}
    \preauthor{\centering\large\emph}
  \postauthor{\par}
      \predate{\centering\large\emph}
  \postdate{\par}
    \date{May 31, 2019}


\begin{document}
\maketitle

This directory contains key results from the bioenergetics analysis of
the impacts of incentive programs on meeting shorebird population
objectives, in each shorebird non-breeding season from 2013-14 through
2016-17.

\subsubsection{habitat\_daily\_stats.csv}\label{habitat_daily_stats.csv}

This file contains estimates of the daily extent of flooded habitat
available in the Central Valley by land cover type, including separate
estimates for both total open water and accessible water (i.e.,
\textless{}10cm deep and accessible to most shorebirds). The table can
be thought of as 2 parts (one for open water and one for accessible
water), each of which is subdivided into 4 parts (one for each
non-breeding season).

\textbf{Description of fields:}

\begin{itemize}
\tightlist
\item
  \textbf{group:} label identifying shorebird non-breeding season; 4
  options: 2013-14, 2014-15, 2015-16, 2016-17
\item
  \textbf{time:} integer identifying the day of the shorebird
  non-breeding season, where day 1 = July 1; ranges 1 - 319
\item
  \textbf{corn:} estimated hectares of fields planted with corn that are
  flooded on each day
\item
  \textbf{other:} estimated hectares of fields planted with ``other''
  suitable crops that are flooded on each day; see Dybala et al. 2017
  SFEWS for a list of ``other'' crops
\item
  \textbf{perm:} estimated hectares of permanent/semi-permanent managed
  wetlands that are flooded on each day
\item
  \textbf{rice:} estimated hectares of fields planted with rice that are
  flooded on each day
\item
  \textbf{seas:} estimated hectares of seasonal managed wetlands that
  are flooded on each day
\item
  \textbf{br\_fall:} estimated hectares of fields enrolled in fall Bird
  Returns program and flooded on each day
\item
  \textbf{br\_spring:} estimated hectares of fields enrolled in spring
  (or winter) Bird Returns program and flooded on each day
\item
  \textbf{whep\_fall:} estimated hectares of fields enrolled in WHEP
  fall flooding practice and currently flooded on each day
\item
  \textbf{whep\_vardd:} estimated hectares of fields enrolled in WHEP
  variable drawdown practice and currently flooded on each day
\item
  \textbf{watertype:} label identifying whether the area flooded
  represents total open water (open) or only shallow water accessible to
  shorebirds (accessible); 2 options: open or accessible
\end{itemize}

\subsubsection{bioenergetics\_results\_energy.csv}\label{bioenergetics_results_energy.csv}

This file contains the main results of the bioenergetics analysis,
including the daily energy requirements (kJ) of the shorebird
population, the daily energy supply available, and the energy shortfall
as the difference between the two. This table can be thought of as 2
parts (one for the shorebird population objectives and one for the
baseline population size), each of which is subdivided into 3 parts
representing different scenarios: including all incentive programs,
excluding all incentive programs, and including only the fall incentive
programs (i.e.~br\_fall and whep\_fall). Each of these individual
scenarios is further subdivided into 4 parts representing each
non-breeding season.

\textbf{Description of fields:}

\begin{itemize}
\tightlist
\item
  \textbf{scenario:} label identifying which combination of population
  objectives and incentive scenario is represented; 6 options:

  \begin{itemize}
  \tightlist
  \item
    obj\_det: population objectives, all incentives
  \item
    obj\_det2: population objectives, no incentives
  \item
    obj\_det3: population objectives, only fall incentives
  \item
    obs\_det: baseline population, all incentives
  \item
    obs\_det2: baseline population, no incentives
  \item
    obs\_det3: baseline population, only fall incentives
  \end{itemize}
\item
  \textbf{group:} label identifying shorebird non-breeding season; 4
  options: 2013-14, 2014-15, 2015-16, 2016-17
\item
  \textbf{time:} integer identifying the day of the shorebird
  non-breeding season, where day 1 = July 1; ranges 1 - 319
\item
  \textbf{DER:} daily energy requirement of the shorebird population
\item
  \textbf{supply:} daily total energy supply across all flooded habitats
  included in the scenario
\item
  \textbf{accessible:} daily total energy supply in accessible flooded
  habitats
\item
  \textbf{shortfall:} daily energy shortfall, calculated as the
  difference between DER and accessible; values \textgreater{} 0
  indicate an energy shortfall
\item
  \textbf{incentives:} label identifying which incentive programs are
  included in the scenario, corresponds to the scenario field above; 3
  options: all, none, and fall\_only
\item
  \textbf{population:} label identifying which population objectives are
  included in the scenario, corresponds to the scenario field above; 2
  options: objectives or baseline
\end{itemize}

\subsubsection{bioenergetics\_energy\_consumed.csv}\label{bioenergetics_energy_consumed.csv}

This file provides further insights into the bioenergetics analysis,
showing the daily amount of energy (kJ) estimated to be consumed in each
land cover type. This amount reflects the assumption of an ideal free
distribution of the shorebird daily energy requirements across all
accessible flooded habitat available on that day, according to the
estimated average energy density remaining in each habitat. (However,
the daily energy consumed in each land cover type cannot exceed the
total amount of energy accessible in each land cover type.) This table
can be used to examine the relative contributions of each land cover
type to the total shorebird energy needs over the entire non-breeding
season. As above, this table can be thought of as 2 parts (one for the
shorebird population objectives and one for the baseline population
size), each of which is subdivided into 3 parts representing different
scenarios: including all incentive programs, excluding all incentive
programs, and including only the fall incentive programs (i.e.~br\_fall
and whep\_fall). Each of these individual scenarios is further
subdivided into 4 parts representing each non-breeding season.

\textbf{Description of fields:}

\begin{itemize}
\tightlist
\item
  \textbf{scenario:} label identifying which combination of population
  objectives and incentive scenario is represented; 6 options:

  \begin{itemize}
  \tightlist
  \item
    obj\_det: population objectives, all incentives
  \item
    obj\_det2: population objectives, no incentives
  \item
    obj\_det3: population objectives, only fall incentives
  \item
    obs\_det: baseline population, all incentives
  \item
    obs\_det2: baseline population, no incentives
  \item
    obs\_det3: baseline population, only fall incentives
  \end{itemize}
\item
  \textbf{group:} label identifying shorebird non-breeding season; 4
  options: 2013-14, 2014-15, 2015-16, 2016-17
\item
  \textbf{time:} integer identifying the day of the shorebird
  non-breeding season, where day 1 = July 1; ranges 1 - 319
\item
  \textbf{corn:} estimated energy consumed in fields planted with corn
  that are flooded on each day
\item
  \textbf{other:} estimated energy consumed in fields planted with
  ``other'' suitable crops that are flooded on each day
\item
  \textbf{perm:} estimated energy consumed in permanent/semi-permanent
  managed wetlands that are flooded on each day
\item
  \textbf{rice:} estimated energy consumed in fields planted with rice
  that are flooded on each day
\item
  \textbf{seas:} estimated energy consumed in seasonal managed wetlands
  that are flooded on each day
\item
  \textbf{br\_fall:} estimated energy consumed in fields enrolled in
  fall Bird Returns program and flooded on each day
\item
  \textbf{br\_spring:} estimated energy consumed in fields enrolled in
  spring (or winter) Bird Returns program and flooded on each day
\item
  \textbf{whep\_fall:} estimated energy consumed in fields enrolled in
  WHEP fall flooding practice and currently flooded on each day
\item
  \textbf{whep\_vardd:} estimated energy consumed in fields enrolled in
  WHEP variable drawdown practice and currently flooded on each day
\item
  \textbf{incentives:} label identifying which incentive programs are
  included in the scenario, corresponds to the scenario field above; 3
  options: all, none, and fall\_only
\item
  \textbf{population:} label identifying which population objectives are
  included in the scenario, corresponds to the scenario field above; 2
  options: objectives or baseline
\end{itemize}


\end{document}
